\capitulo{1}{Introducción}

Encontrar información en las páginas de interés público es, a veces, una tarea complicada. Portales web que proporcionan información a un amplio sector de la ciudadanía deberían ser simples y estar estructurados correctamente.

La página web de la Universidad de Burgos es una de las mejores organizadas en comparación con otras universidades tanto nacionales como internacionales. Sin embargo, como denominador común, todas ellas tienen un defecto comprensible, disponen de demasiada información. En ocasiones no es posible encontrar de manera rápida y precisa un apartado concreto del sitio web y hay que sumergirse en ciertos menús que pueden llegar a ser abstractos. La Universidad de Burgos no es la excepción, a través de los menús de su página principal es posible acceder a todos los recursos, sin embargo esto genera un gran problema, son necesarios una cantidad importante de menús que en ocasiones no son todo lo concretos que desearíamos.

El diseño del portal web de la Universidad de Burgos está estructurado en cinco menús generales que contienen una cantidad de submenús que puede parecer excesiva. Además, cabe destacar la existencia de un problema de duplicación de apartados, es decir, a través de distintos submenús puedes llegar a un mismo recurso. Estas situaciones pueden ocasionar desorientación en un usuario.

Actualmente la página web de la Universidad de Burgos dispone de un apartado de búsqueda que permite a los usuarios introducir palabras clave para encontrar los apartados deseados.

Este buscador proporcionado es verdaderamente efectivo si el usuario proporciona palabras clave en el proceso de búsqueda, pero su efectividad baja al expresar con un lenguaje natural la búsqueda deseada.

El método propuesto para facilitar la navegación por la página de la Universidad de Burgos pretende otorgar al usuario una interfaz disponible en todo momento y con aspecto de chat, sobre la que poder realizar una búsqueda. 

El asistente, llamado UBUassistant, se encargará de buscar una respuesta al texto introducido por el usuario mediante algoritmos de minería de datos. Con este asistente no será necesario expresarse de una manera artificial utilizando palabras clave, sino que será el propio asistente el encargado de analizar una frase expresada de manera natural para encontrar la respuesta adecuada.

La funcionalidad de UBUassistant se extiende hasta proporcionar sugerencias cuando encuentra múltiples respuestas, además de cuando no existe una contestación. La aplicación proporciona estadísticas de uso, pudiendo obtener los casos más buscados e incluso realizar un aprendizaje de manera supervisada.

\section{Estructura de la memoria}\label{estructura-de-la-memoria}
La memoria tiene la siguiente división en apartados:

\begin{itemize}
\tightlist
\item
  \textbf{Introducción:} En este apartado se realiza una descripción de una manera breve del problema que se intenta resolver y la solución otorgada. Además incluye subapartados con la estructura de la memoria y el listado de materiales adjuntos.
\item
  \textbf{Objetivos del proyecto:} sección donde se explican los objetivos de desarrollar un proyecto de estas características.
\item
  \textbf{Conceptos teóricos:} capítulo en el que se abordan los conceptos teóricos necesarios para comprender el resultado final del proyecto.
\item
  \textbf{Técnicas y herramientas:} en esta sección se describen las herramientas y las técnicas que se han utilizado para el desarrollo y gestión del proceso del proyecto.
\item
  \textbf{Aspectos relevantes del desarrollo:} apartado donde se tratan aquellos aspectos que se consideran destacados en el desarrollo del proyecto.
\item
  \textbf{Trabajos relacionados:} capitulo que expone y describe aquellos trabajos que están relacionados con la temática de asistente virtual.
\item
  \textbf{Conclusiones y líneas de trabajo futuras:} sección que explica las conclusiones obtenidas tras la realización del proyecto y la funcionalidad que es posible añadir en el futuro.
\end{itemize}

Además, se proporcionan los siguientes anexos:

\begin{itemize}
\tightlist
\item
  \textbf{Plan del proyecto software:} capítulo donde se expone planificación temporal del proyecto y su viabilidad.
\item
  \textbf{Especificación de requisitos:} en este apartado se desarrollan los objetivos del software y la especificación de requisitos.
\item
  \textbf{Especificación de diseño:} sección que describe el diseño de datos, el diseño procedimental y el diseño arquitectónico.
\item
  \textbf{Documentación técnica de programación:} en este capítulo de explica todo lo relacionado con la programación, la estructura de directorios, el manual del programador y las pruebas realizadas.
\item
  \textbf{Documentación de usuario:} apartado que realiza un explicación sobre los requisitos de usuarios, la instalación y proporciona un manual de usuario.
\end{itemize}

\section{Materiales adjuntos}\label{materiales-adjuntos}

Los materiales que se adjuntan con la memoria son: 

\begin{itemize}
\tightlist
\item
	Aplicación Java UBUassistant.
\item	
	JavaDoc.
\end{itemize}

Además, los siguientes recursos están accesibles a través de internet:

\begin{itemize}
\tightlist
\item
  Repositorio del proyecto.
\end{itemize}


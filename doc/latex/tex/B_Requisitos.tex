\apendice{Especificación de Requisitos}

\section{Introducción}

La especificación de requisitos del software consiste en realizar una descripción completa del comportamiento del sistema que va a desarrollarse. Puede verse como un contrato entre los desarrolladores y el cliente.

En esta especificación se incluyen un conjunto de casos de uso donde se detallan los pasos o actividades de un usuario de la aplicación para llevar a cabo un determinado proceso.

Se recomienda el uso de un lenguaje cotidiano e informal para que el documento obtenido sea comprendido fácilmente por todos los involucrados en el desarrollo del sistema.

Se pueden identificar varios tipos de requisitos, centrándonos en los requisitos funcionales y requisitos no funcionales \cite{wiki:ers}.

\begin{itemize}
\tightlist
\item
  \textbf{Requisitos funcionales:} determinan los servicios que el sistema debe proporcionar. Están relacionados con los casos de uso.
\item
  \textbf{Requisitos no funcionales:}  determinan cómo debe ser el sistema imponiendo restricciones de diseño o implementación.
\end{itemize}

Existen una serie de características que hacen que una especificación de requisitos del software disponga de una calidad adecuada. Estas características están definidas por el estándar IEEE 830-1998, el cual determina que una buena especificación de requisitos debe ser \cite{wiki:ers}:

\begin{itemize}
\tightlist
\item
  \textbf{Completa:} se deben reflejar todos los requerimientos y definir correctamente sus relaciones.
\item
  \textbf{Consistente:} no deben existir incoherencias entre los requerimientos.
\item
  \textbf{Inequívoca:} se debe usar un lenguaje sencillo para no dar lugar a equivocaciones.
\item
  \textbf{Correcta:} el producto debe cumplir todos los requerimientos.
\item
  \textbf{Trazable:} se deben identificar de forma única los requerimientos.
\item
  \textbf{Priorizable:}  se deben poder organizar los requisitos según su importancia.
\item
  \textbf{Modificable:} cualquier requerimiento debe ser modificable de forma sencilla.
\item
  \textbf{Verificable:} se debe poder probar mediante algún método.
\end{itemize}


\section{Objetivos generales}

\section{Catalogo de requisitos}

\section{Especificación de requisitos}



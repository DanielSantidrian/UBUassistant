\apendice{Documentación técnica de programación}

\section{Introducción}

En la documentación técnica del programador se explica la estructura final de directorios de la aplicación desarrollada, así como los pasos para compilar, instalar y ejecutar el proyecto. Además se incluye un manual para el programador y se lleva a cabo una explicación de las pruebas realizadas al sistema.

\section{Estructura de directorios}

El proyecto consta de los siguientes directorios:

\begin{itemize}
\tightlist
\item \textbf{/:} el directorio raíz del proyecto contiene el fichero de la licencia, el archivo README y el fichero que almacena las dependencias que se cargarán dinámicamente mediante \emph{Maven}. Además, se encuentran todas las carpetas que se especifican a continuación.
\item \textbf{/doc:} documentación del proyecto.
\item \textbf{/doc/javadoc:} documentación del código generada por JavaDoc.
\item \textbf{/doc/latex:} documentación generada mediante \emph{LaTeX}.
\item \textbf{/doc/latex/img:} imágenes utilizadas en la documentación.
\item \textbf{/doc/latex/tex:} ficheros utilizados en para generar la documentación \emph{LaTeX}.
\item \textbf{/lib:} archivos binarios y recursos del algoritmo de inteligencia artificial para añadir al \emph{path} del proyecto y hacerlo funcionar.
\item \textbf{/rsc:} recursos del proyecto, como el fichero de creación de base de datos y los distintos ejecutables para realizar las pruebas con \emph{Selenium}.
\item \textbf{/src:} código fuente de la aplicación web.
\item \textbf{/src/main:} clases, recursos y páginas HTML/JSP del proyecto.
\item \textbf{/src/main/java:} clases Java del proyecto.
\item \textbf{/src/main/resources:} recursos de configuración.
\item \textbf{/src/main/webapp:} ficheros para la página web (HTML, JSP, CSS, JS, Imágenes, XML)
\item \textbf{/src/test/java: } test unitarios, de integración y de interfaz.
\item \textbf{/target:} clases compiladas de nuestro proyecto.
\end{itemize}

\section{Manual del programador}

\section{Compilación, instalación y ejecución del proyecto}

\section{Pruebas del sistema}

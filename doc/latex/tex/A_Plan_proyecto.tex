\apendice{Plan de Proyecto Software}

\section{Introducción}

La planificación de un proyecto es una de las tareas fundamentales antes y durante el desarrollo del mismo. La planificación proporciona información necesaria que permite controlar el proyecto. Para recoger esta información se lleva a cabo un análisis sobre cada una de las partes que forman el proyecto. La fase de planificación se puede dividir en:

\begin{itemize}
\tightlist
\item
  \textbf{Planificación temporal:} consiste en establecer un marco de tiempos donde se recogen las distintas duraciones de cada una de las partes o fases del desarrollo del proyecto. Se establece una fecha de inicio del proyecto y, considerando el tiempo necesario en cada parte del desarrollo, también se establece una fecha de finalización aproximada.
  
\item
  \textbf{Estudio de viabilidad:} consiste en realizar los estudios pertinentes para que el proyecto salga adelante. Es un paso previo necesario al desarrollo del software donde se toma la decisión de llevarlo a cabo. Se diferencian dos etapas en el estudio de la viabilidad:  
  \begin{itemize}
  \tightlist
  \item
  	\textbf{Viabilidad económica:} estudio que estima la rentabilidad de desarrollar el proyecto, analizando los costes y los beneficios.
  \item
  	\textbf{Viabilidad legal:} estudio donde se analiza el marco legal en el ámbito de la aplicabilidad del proyecto. En el caso de los proyectos software cabe considerar la adquisición de licencias y contratos.
  \end{itemize}
\end{itemize}

\section{Planificación temporal}

La metodología principal que se ha intentado aplicar en el desarrollo del proyecto ha sido Scrum.

No se siguieron las indicaciones de la metodología Scrum de manera estricta al tratarse de un proyecto pequeño, sin equipo de codificación grande, y sin poder realizar todas las reuniones necesarias. Sin embargo sí que se siguieron las siguientes pautas:

\begin{itemize}
\tightlist
\item
Desarrollo iterativo e incremental del producto mediante sprints.
\item
Duración en la mayoría de los sprints de una semana, exceptuando alguna excepción por días festivos.
\item
Reuniones al final de cada sprint para evaluar el producto obtenido y planificar la siguiente iteración.
\item
Entrega del producto totalmente funcional al final de cada sprint.
\end{itemize}

\section{Estudio de viabilidad}

\subsection{Viabilidad económica}

\subsection{Viabilidad legal}



\apendice{Plan de Proyecto Software}

\section{Introducción}

La planificación de un proyecto es una de las tareas fundamentales antes y durante el desarrollo del mismo y consiste en determinar todas las acciones a realizar para alcanzar un objetivo determinado. La planificación proporciona información necesaria que permite controlar y reducir los factores de riesgo de proyecto. Para recoger esta información se lleva a cabo un análisis sobre cada una de las partes que forman el proyecto. 

La fase de planificación está compuesta principalmente por:

\begin{itemize}
\tightlist
\item
  \textbf{Planificación temporal:} consiste en establecer un marco de tiempos donde se recogen las distintas duraciones de cada una de las partes o fases del desarrollo del proyecto. Se establece una fecha de inicio del proyecto y, considerando el tiempo necesario en cada parte del desarrollo, también se establece una fecha de finalización aproximada.
  
\item
  \textbf{Estudio de viabilidad:} consiste en realizar los estudios pertinentes para que el proyecto salga adelante. Es un paso previo necesario al desarrollo del software donde se toma la decisión de llevarlo a cabo. Se diferencian dos etapas en el estudio de la viabilidad:  
  \begin{itemize}
  \tightlist
  \item
  	\textbf{Viabilidad económica:} estudio que estima la rentabilidad de desarrollar el proyecto, analizando los costes y los beneficios.
  \item
  	\textbf{Viabilidad legal:} estudio donde se analiza el marco legal en el ámbito de la aplicabilidad del proyecto. En el caso de los proyectos software cabe considerar la adquisición de licencias y contratos.
  \end{itemize}
\end{itemize}

\section{Planificación temporal}

La metodología principal que se ha intentado aplicar en el desarrollo del proyecto ha sido Scrum.

No se siguieron las indicaciones de la metodología Scrum de manera estricta al tratarse de un proyecto pequeño, sin equipo de codificación grande, y sin poder realizar todas las reuniones necesarias. Sin embargo sí que se siguieron las siguientes pautas:

\begin{itemize}
\tightlist
\item
Desarrollo iterativo e incremental del producto mediante sprints.
\item
Duración en la mayoría de los sprints de una semana, exceptuando alguna excepción por días festivos.
\item
Reuniones al final de cada sprint para evaluar el producto obtenido y planificar la siguiente iteración.
\item
Entrega del producto totalmente funcional al final de cada sprint.
\end{itemize}

En cada iteración realizada se han realizado diferentes tareas. Los diferentes sprints se detallan a continuación.

\subsection{Sprint 0 (07/02/17 - 16/02/17)}

En la reunión de planificación del primer sprint se concretaron varios objetivos del proyecto. Se profundizó sobre el producto final que se quería obtener.

Los objetivos que se marcaron en este sprint fueron: la profundización en el producto final a desarrollar, especificación del lenguaje de programación a utilizar así como determinar el conjunto de herramientas para el desarrollo del producto, el sistema de comunicación, el repositorio para el control de versiones y la herramienta de gestión de proyectos.

Las tareas que se llevaron a cabo para cumplir con los objetivos del sprint pueden verse en \href{https://github.com/DanielSantidrian/UBUassistant/milestone/3?closed=1}{Sprint0}

Las horas estimadas fueron 2,25h, invirtiendo finalmente 1,70h completando la totalidad de las tareas planteadas.

\imagen{burndowns/sprint0}{Gráfico burndown sprint 0.}

\subsection{Sprint 1 (14/02/17 - 20/02/17)}

En este sprint se realizaron tareas de documentación, como la selección del framework principal para desarrollar el algoritmo de inteligencia artificial. Una vez elegido se realizó una primera aproximación de la aplicación. Además se buscaron ejemplos del proyecto y se realizó una aproximación de los requisitos del producto final.

Además, sobre la primera aproximación de la aplicación, una vez enseñada al tutor, se realizaron varios cambios de funcionalidad, destacando el análisis de todas las palabras introducidas por el usuario, el cambio en la forma de representar los casos de una frase a 3 palabras clave y la inclusión del soporte para leer los casos de una base de datos.

Las tareas que se llevaron a cabo para cumplir con los objetivos del sprint pueden verse en \href{https://github.com/DanielSantidrian/UBUassistant/milestone/4?closed=1}{Sprint1}

Las horas estimadas fueron 11,25h, invirtiendo finalmente 22,75h completando la totalidad de las tareas planteadas.

\imagen{burndowns/sprint1}{Gráfico burndown sprint 1.}

\subsection{Sprint 2 (21/02/17 - 27/02/17)}

Los objetivos de este sprint fueron principalmente transformar la primera aproximación, sobre la que también se trabajó en el anterior sprint, en una versión más cercana a los requisitos funcionales planteados.

Las tareas que se llevaron a cabo para cumplir con los objetivos del sprint pueden verse en \href{https://github.com/DanielSantidrian/UBUassistant/milestone/5?closed=1}{Sprint2}

Las horas estimadas fueron 5,50h, invirtiendo finalmente 7,25h completando la totalidad de las tareas planteadas.

\imagen{burndowns/sprint2}{Gráfico burndown sprint 2.}

\subsection{Sprint 3 (28/02/17 - 06/03/17)}

Los objetivos marcados para este sprint fueron seguir desarrollando nuevos requisitos funcionales y mejorar el sistema y su interfaz gráfica.

Las tareas que se llevaron a cabo para cumplir con los objetivos del sprint pueden verse en \href{https://github.com/DanielSantidrian/UBUassistant/milestone/6?closed=1}{Sprint3}

Las horas estimadas fueron 6,50h, invirtiendo finalmente 12,25h completando la totalidad de las tareas planteadas.

\imagen{burndowns/sprint3}{Gráfico burndown sprint 3.}

\subsection{Sprint 4 (07/03/17 - 13/03/17)}

En este sprint se continuó con la adición de requisitos, en este caso se añadieron los sistemas de recogida de estadísticas y el sistema de aprendizaje supervisado.

Las tareas que se llevaron a cabo para cumplir con los objetivos del sprint pueden verse en \href{https://github.com/DanielSantidrian/UBUassistant/milestone/7?closed=1}{Sprint4}

Las horas estimadas fueron 4,75h, invirtiendo finalmente 6,75h completando la totalidad de las tareas planteadas.

\imagen{burndowns/sprint4}{Gráfico burndown sprint 4.}

\subsection{Sprint 5 (14/03/17 - 20/03/17)}

El objetivo de este sprint fue cambiar la estructura de la tabla donde se almacenan las estadísticas para guardar más información relevante.

Las tareas que se llevaron a cabo para cumplir con los objetivos del sprint pueden verse en \href{https://github.com/DanielSantidrian/UBUassistant/milestone/8?closed=1}{Sprint5}

Las horas estimadas fueron 1,50h, invirtiendo finalmente 2,25h completando la totalidad de las tareas planteadas.

\imagen{burndowns/sprint5}{Gráfico burndown sprint 5.}

\subsection{Sprint 6 (21/03/17 - 17/04/17)}

Los objetivos marcados en este sprint se centraron en incluir nuevas características al producto, como ofrecer la posibilidad de valorar la respuesta cuando hay varias disponibles. Además se mejoraron ciertas funcionalidades de las que ya se disponía, como la búsqueda de respuesta y el almacenamiento de estadísticas. También se modificó la estructura de clases separando la interfaz gráfica del algoritmo de búsqueda de respuestas.

Las tareas que se llevaron a cabo para cumplir con los objetivos del sprint pueden verse en \href{https://github.com/DanielSantidrian/UBUassistant/milestone/9?closed=1}{Sprint6}

Las horas estimadas fueron 9,75h, invirtiendo finalmente 12,25h completando la totalidad de las tareas planteadas.

\imagen{burndowns/sprint6}{Gráfico burndown sprint 6.}

\subsection{Sprint 7 (18/04/17 - 01/05/17)}

En este sprint se marcó como objetivo principal transformar el proyecto Java en una aplicación web mediante JSP. El proceso de transformación no alteró el algoritmo de búsqueda de respuestas, sin embargo, sí que supuso esfuerzo el cambio en la interacción con el usuario mediante la interfaz gráfica.

Además de este cambio, también se marcó como objetivo completar la base de datos con todos los casos que de forma genérica se pueden obtener de la página principal de la Universidad de Burgos.

Las tareas que se llevaron a cabo para cumplir con los objetivos del sprint pueden verse en \href{https://github.com/DanielSantidrian/UBUassistant/milestone/10?closed=1}{Sprint7}

Las horas estimadas fueron 16,75h, invirtiendo finalmente 26,75h completando la totalidad de las tareas planteadas.

\imagen{burndowns/sprint7}{Gráfico burndown sprint 7.}

\subsection{Sprint 8 (02/05/17 - 09/05/17)}

Los objetivos de este sprint se centraron principalmente en la documentación del proyecto, avanzando en la realización de la memoria y añadiendo los comentarios de JavaDoc al código, además se eligieron las herramientas para realizar la propia documentación así como las herramientas para llevar a cabo las pruebas.

En cuanto a funcionalidad, se implementaron requisitos como la forma de iniciar la aplicación y arreglar un \emph{bug} del sistema a la hora de mostrar recomendaciones cuando hay varias respuestas.

Las tareas que se llevaron a cabo para cumplir con los objetivos del sprint pueden verse en \href{https://github.com/DanielSantidrian/UBUassistant/milestone/11?closed=1}{Sprint8}

Las horas estimadas fueron 6,25h, invirtiendo finalmente 9,25h completando la totalidad de las tareas planteadas.

\imagen{burndowns/sprint8}{Gráfico burndown sprint 8.}

\subsection{Sprint 9 (10/05/17 - 18/05/17)}

En este sprint los objetivos se centraron en realizar una primera versión de la página de administración de la aplicación, corregir bugs y la realización de pruebas unitarias.

Las tareas que se llevaron a cabo para cumplir con los objetivos del sprint pueden verse en \href{https://github.com/DanielSantidrian/UBUassistant/milestone/12?closed=1}{Sprint9}

Las horas estimadas fueron 13h, invirtiendo finalmente 19,75h quedando una tarea pendiente para el siguiente sprint.

\imagen{burndowns/sprint9}{Gráfico burndown sprint 9.}

\subsection{Sprint 10 (19/05/17 - 31/05/17)}

Los objetivos de este sprint fueron añadir funcionalidad a la página de administración teniendo en cuenta los requisitos.

Las tareas que se llevaron a cabo para cumplir con los objetivos del sprint pueden verse en \href{https://github.com/DanielSantidrian/UBUassistant/milestone/13?closed=1}{Sprint10}

Las horas estimadas fueron 6h, invirtiendo finalmente 9,50h quedando una tarea pendiente para el siguiente sprint.

\imagen{burndowns/sprint10}{Gráfico burndown sprint 10.}

\subsection{Sprint 11 (01/06/17 - 07/06/17)}

En este sprint se fijaron objetivos para adaptar la interfaz web a distintas resoluciones de pantallas y móviles, realizar pruebas de interfaz y añadir pequeñas funcionalidades.

Las tareas que se llevaron a cabo para cumplir con los objetivos del sprint pueden verse en \href{https://github.com/DanielSantidrian/UBUassistant/milestone/14?closed=1}{Sprint11}

Las horas estimadas fueron 12,75h, invirtiendo finalmente 17h quedando una tarea pendiente para el siguiente sprint.

\imagen{burndowns/sprint11}{Gráfico burndown sprint 11.}

\subsection{Resumen}\label{resumen}

La siguiente tabla recoge el desglose de las horas dedicadas por tipo de tarea a modo de resumen.

\begin{longtable}[]{@{}lrr@{}}
\toprule
\begin{minipage}[b]{0.37\columnwidth}\raggedright\strut
Categoría\strut
\end{minipage} & \begin{minipage}[b]{0.19\columnwidth}\raggedright\strut
\emph{Issues}\strut
\end{minipage} & \begin{minipage}[b]{0.19\columnwidth}\raggedright\strut
Horas\strut
\end{minipage}\tabularnewline
\midrule
\endhead

\begin{minipage}[t]{0.37\columnwidth}\raggedright\strut
\emph{Documentación}\strut
\end{minipage} & \begin{minipage}[t]{0.19\columnwidth}\raggedright\strut
19\strut
\end{minipage} & \begin{minipage}[t]{0.19\columnwidth}\raggedright\strut
10,70\strut
\end{minipage}\tabularnewline


\begin{minipage}[t]{0.37\columnwidth}\raggedright\strut
\emph{Test}\strut
\end{minipage} & \begin{minipage}[t]{0.19\columnwidth}\raggedright\strut
3\strut
\end{minipage} & \begin{minipage}[t]{0.19\columnwidth}\raggedright\strut
7\strut
\end{minipage}\tabularnewline


\begin{minipage}[t]{0.37\columnwidth}\raggedright\strut
\emph{Bug}\strut
\end{minipage} & \begin{minipage}[t]{0.19\columnwidth}\raggedright\strut
12\strut
\end{minipage} & \begin{minipage}[t]{0.19\columnwidth}\raggedright\strut
15,25\strut
\end{minipage}\tabularnewline


\begin{minipage}[t]{0.37\columnwidth}\raggedright\strut
\emph{Feature}\strut
\end{minipage} & \begin{minipage}[t]{0.19\columnwidth}\raggedright\strut
63\strut
\end{minipage} & \begin{minipage}[t]{0.19\columnwidth}\raggedright\strut
124\strut
\end{minipage}\tabularnewline
\midrule

\begin{minipage}[t]{0.37\columnwidth}\raggedright\strut
TOTAL\strut
\end{minipage} & \begin{minipage}[t]{0.19\columnwidth}\raggedright\strut
97\strut
\end{minipage} & \begin{minipage}[t]{0.19\columnwidth}\raggedright\strut
156,95\strut
\end{minipage}\tabularnewline
\bottomrule
\caption{Resumen horas dedicadas al proyecto.}
\end{longtable}

\imagen{ResumenHoras}{Porcentaje de horas por tipo de tarea.}


\newpage

\section{Estudio de viabilidad}

\subsection{Viabilidad económica}

El estudio de la viabilidad económica estima la rentabilidad de desarrollar el proyecto, analizando los costes y los beneficios.

Se realiza el estudio suponiendo que el proyecto se lleva a cabo por una empresa.

\subsubsection{Costes humanos}

Se considera un tiempo de contratación similar al disponible por el alumno para realizar el proyecto, alrededor de cinco meses.

En la siguiente tabla se desglosan los costes para un empresa de contratar a un empleado a tiempo completo durante los cinco meses correspondientes. Se han tenido en cuenta los pagos de IRPF (15\%) \cite{pago:irpf} y los pagos a la seguridad social (29,9\%) \cite{pago:ss}.

Para calcular los pagos a la seguridad social por parte de la empresa se han tenido en cuenta las retribuciones comunes (23,60\%), el desempleo de tipo general (5,50\%), el fondo de garantía salarial (0,20\%) y la formación profesional (0,60\%).

\begin{longtable}[]{@{}lr@{}}
\toprule
\begin{minipage}[b]{0.38\columnwidth}\raggedright\strut
\textbf{Concepto}\strut
\end{minipage} & \begin{minipage}[b]{0.20\columnwidth}\raggedright\strut
\textbf{Coste}\strut
\end{minipage}\tabularnewline
\midrule
\endhead

\begin{minipage}[t]{0.38\columnwidth}\raggedright\strut
Salario mensual bruto\strut
\end{minipage} & \begin{minipage}[t]{0.20\columnwidth}\raggedright\strut
1200 \euro{}\strut
\end{minipage}\tabularnewline

\begin{minipage}[t]{0.38\columnwidth}\raggedright\strut
Retención IRPF (15\%)\strut
\end{minipage} & \begin{minipage}[t]{0.20\columnwidth}\raggedright\strut
180 \euro{}\strut
\end{minipage}\tabularnewline

\begin{minipage}[t]{0.38\columnwidth}\raggedright\strut
Seguridad Social (29,9\%)\strut
\end{minipage} & \begin{minipage}[t]{0.20\columnwidth}\raggedright\strut
358,8 \euro{}\strut
\end{minipage}\tabularnewline

\begin{minipage}[t]{0.38\columnwidth}\raggedright\strut
Total bruto\strut
\end{minipage} & \begin{minipage}[t]{0.20\columnwidth}\raggedright\strut
1738,8 \euro{}\strut
\end{minipage}\tabularnewline

\midrule

\begin{minipage}[t]{0.38\columnwidth}\raggedright\strut
\textbf{Total 5 meses}\strut
\end{minipage} & \begin{minipage}[t]{0.20\columnwidth}\raggedright\strut
8694 \euro{}\strut
\end{minipage}\tabularnewline
\bottomrule
\caption{Costes humanos.}
\end{longtable}

\subsubsection{Costes hardware}

Los costes hardware engloban todos aquellos costes de componentes físicos que son necesarios para desarrollar el proyecto.

\begin{longtable}[]{@{}lr@{}}
\toprule
\begin{minipage}[b]{0.38\columnwidth}\raggedright\strut
\textbf{Concepto}\strut
\end{minipage} & \begin{minipage}[b]{0.20\columnwidth}\raggedright\strut
\textbf{Coste}\strut
\end{minipage}\tabularnewline
\midrule
\endhead

\begin{minipage}[t]{0.38\columnwidth}\raggedright\strut
Ordenador Personal \cite{pago:pc} \strut
\end{minipage} & \begin{minipage}[t]{0.20\columnwidth}\raggedright\strut
700 \euro{}\strut
\end{minipage}\tabularnewline

\begin{minipage}[t]{0.38\columnwidth}\raggedright\strut
Servidor Tomcat \cite{tom:server} \strut
\end{minipage} & \begin{minipage}[t]{0.20\columnwidth}\raggedright\strut
100 \euro{}\strut
\end{minipage}\tabularnewline

\midrule

\begin{minipage}[t]{0.38\columnwidth}\raggedright\strut
\textbf{Total}\strut
\end{minipage} & \begin{minipage}[t]{0.20\columnwidth}\raggedright\strut
800 \euro{}\strut
\end{minipage}\tabularnewline
\bottomrule
\caption{Costes hardware.}
\end{longtable}

\subsubsection{Costes licencias}

En los costes de licencias se tiene en cuenta todo el software de pago necesario en el proyecto.

\begin{longtable}[]{@{}lr@{}}
\toprule
\begin{minipage}[b]{0.38\columnwidth}\raggedright\strut
\textbf{Concepto}\strut
\end{minipage} & \begin{minipage}[b]{0.20\columnwidth}\raggedright\strut
\textbf{Coste}\strut
\end{minipage}\tabularnewline
\midrule
\endhead

\begin{minipage}[t]{0.38\columnwidth}\raggedright\strut
Windows 10 Home \cite{pago:w10} \strut
\end{minipage} & \begin{minipage}[t]{0.20\columnwidth}\raggedright\strut
135 \euro{}\strut
\end{minipage}\tabularnewline

\begin{minipage}[t]{0.38\columnwidth}\raggedright\strut
Microsoft Office 2016 \cite{pago:office} \strut
\end{minipage} & \begin{minipage}[t]{0.20\columnwidth}\raggedright\strut
279 \euro{}\strut
\end{minipage}\tabularnewline

\midrule

\begin{minipage}[t]{0.38\columnwidth}\raggedright\strut
\textbf{Total}\strut
\end{minipage} & \begin{minipage}[t]{0.20\columnwidth}\raggedright\strut
414 \euro{}\strut
\end{minipage}\tabularnewline
\bottomrule
\caption{Costes licencias.}
\end{longtable}

\subsubsection{Costes redes y comunicación}

Se estiman los costes de implantar una red de comunicación.

\begin{longtable}[]{@{}lr@{}}
\toprule
\begin{minipage}[b]{0.38\columnwidth}\raggedright\strut
\textbf{Concepto}\strut
\end{minipage} & \begin{minipage}[b]{0.20\columnwidth}\raggedright\strut
\textbf{Coste}\strut
\end{minipage}\tabularnewline
\midrule
\endhead

\begin{minipage}[t]{0.38\columnwidth}\raggedright\strut
Internet \cite{pago:int} \strut
\end{minipage} & \begin{minipage}[t]{0.20\columnwidth}\raggedright\strut
150 \euro{}\strut
\end{minipage}\tabularnewline

\midrule

\begin{minipage}[t]{0.38\columnwidth}\raggedright\strut
\textbf{Total}\strut
\end{minipage} & \begin{minipage}[t]{0.20\columnwidth}\raggedright\strut
150 \euro{}\strut
\end{minipage}\tabularnewline
\bottomrule
\caption{Costes redes y comunicación.}
\end{longtable}

\subsubsection{Costes infraestructura}

En esta sección se tienen en cuenta los costes de alojamiento.

\begin{longtable}[]{@{}lr@{}}
\toprule
\begin{minipage}[b]{0.38\columnwidth}\raggedright\strut
\textbf{Concepto}\strut
\end{minipage} & \begin{minipage}[b]{0.20\columnwidth}\raggedright\strut
\textbf{Coste}\strut
\end{minipage}\tabularnewline
\midrule
\endhead

\begin{minipage}[t]{0.38\columnwidth}\raggedright\strut
Alquiler oficina \cite{pago:alq} \strut
\end{minipage} & \begin{minipage}[t]{0.20\columnwidth}\raggedright\strut
1000 \euro{}\strut
\end{minipage}\tabularnewline

\midrule

\begin{minipage}[t]{0.38\columnwidth}\raggedright\strut
\textbf{Total}\strut
\end{minipage} & \begin{minipage}[t]{0.20\columnwidth}\raggedright\strut
1000 \euro{}\strut
\end{minipage}\tabularnewline
\bottomrule
\caption{Costes alojamiento.}
\end{longtable}

\subsubsection{Costes impresión}

En esta sección se tienen en cuenta los costes de impresión de la documentación.

\begin{longtable}[]{@{}lr@{}}
\toprule
\begin{minipage}[b]{0.38\columnwidth}\raggedright\strut
\textbf{Concepto}\strut
\end{minipage} & \begin{minipage}[b]{0.20\columnwidth}\raggedright\strut
\textbf{Coste}\strut
\end{minipage}\tabularnewline
\midrule
\endhead

\begin{minipage}[t]{0.38\columnwidth}\raggedright\strut
Impresión memoria y cartel\strut
\end{minipage} & \begin{minipage}[t]{0.20\columnwidth}\raggedright\strut
30 \euro{}\strut
\end{minipage}\tabularnewline

\midrule

\begin{minipage}[t]{0.38\columnwidth}\raggedright\strut
\textbf{Total}\strut
\end{minipage} & \begin{minipage}[t]{0.20\columnwidth}\raggedright\strut
30 \euro{}\strut
\end{minipage}\tabularnewline
\bottomrule
\caption{Costes impresión.}
\end{longtable}

\subsubsection{Costes totales}

Los costes totales de llevar a cabo el proyecto por una empresa son:

\begin{longtable}[]{@{}lr@{}}
\toprule
\begin{minipage}[b]{0.38\columnwidth}\raggedright\strut
\textbf{Tipo coste}\strut
\end{minipage} & \begin{minipage}[b]{0.20\columnwidth}\raggedright\strut
\textbf{Coste}\strut
\end{minipage}\tabularnewline
\midrule
\endhead

\begin{minipage}[t]{0.38\columnwidth}\raggedright\strut
Humano\strut
\end{minipage} & \begin{minipage}[t]{0.20\columnwidth}\raggedright\strut
8694 \euro{}\strut
\end{minipage}\tabularnewline

\begin{minipage}[t]{0.38\columnwidth}\raggedright\strut
Hardware\strut
\end{minipage} & \begin{minipage}[t]{0.20\columnwidth}\raggedright\strut
800 \euro{}\strut
\end{minipage}\tabularnewline

\begin{minipage}[t]{0.38\columnwidth}\raggedright\strut
Licencias\strut
\end{minipage} & \begin{minipage}[t]{0.20\columnwidth}\raggedright\strut
414 \euro{}\strut
\end{minipage}\tabularnewline

\begin{minipage}[t]{0.38\columnwidth}\raggedright\strut
Redes y comunicación\strut
\end{minipage} & \begin{minipage}[t]{0.20\columnwidth}\raggedright\strut
150 \euro{}\strut
\end{minipage}\tabularnewline

\begin{minipage}[t]{0.38\columnwidth}\raggedright\strut
Infraestructura\strut
\end{minipage} & \begin{minipage}[t]{0.20\columnwidth}\raggedright\strut
1000 \euro{}\strut
\end{minipage}\tabularnewline

\begin{minipage}[t]{0.38\columnwidth}\raggedright\strut
Impresión\strut
\end{minipage} & \begin{minipage}[t]{0.20\columnwidth}\raggedright\strut
30 \euro{}\strut
\end{minipage}\tabularnewline

\midrule

\begin{minipage}[t]{0.38\columnwidth}\raggedright\strut
\textbf{Total}\strut
\end{minipage} & \begin{minipage}[t]{0.20\columnwidth}\raggedright\strut
11088 \euro{}\strut
\end{minipage}\tabularnewline
\bottomrule
\caption{Costes totales.}
\end{longtable}

\subsubsection{Beneficios}

El producto desarrollado es gratuito y accesible a través de su página web.

Los beneficios de desarrollar esta aplicación se podrían obtener a través de su venta o implantación a un tercero, obteniendo ingresos por la mantenibilidad del producto. Además indirectamente otorga un valor añadido a la web que lo implanta pudiendo generar más visitas y por ente más dinero.

\subsection{Viabilidad legal}

La viabilidad legal de un proyecto consiste en el estudio del marco  legal en el ámbito de la aplicabilidad del proyecto. En el caso de los proyectos software hay que considerar las licencias de las dependencias usadas.

Hay que comprobar que todas las dependencias y bibliotecas usadas en el proyecto dispongan de licencias permisivas para el uso del código por parte de terceros.

\subsubsection{Software}

La licencias de las dependencias utilizadas se recogen en la siguiente tabla.


\begin{longtable}[]{@{}llll@{}}
\toprule
\begin{minipage}[b]{0.18\columnwidth}\raggedright\strut
Dependencia\strut
\end{minipage} & \begin{minipage}[b]{0.10\columnwidth}\raggedright\strut
Versión\strut
\end{minipage} & \begin{minipage}[b]{0.49\columnwidth}\raggedright\strut
Descripción\strut
\end{minipage} & \begin{minipage}[b]{0.11\columnwidth}\raggedright\strut
Licencia\strut
\end{minipage}\tabularnewline\tabularnewline
\midrule
\endhead


\begin{minipage}[t]{0.18\columnwidth}\raggedright\strut
jCOLIBRI \cite{lic:jcol} \strut
\end{minipage} & \begin{minipage}[t]{0.08\columnwidth}\raggedright\strut
2.3\strut
\end{minipage} & \begin{minipage}[t]{0.49\columnwidth}\raggedright\strut
Framework de inteligencia artificail.\strut
\end{minipage} & \begin{minipage}[t]{0.11\columnwidth}\raggedright\strut
LGPL\strut
\end{minipage}\tabularnewline\tabularnewline


\begin{minipage}[t]{0.18\columnwidth}\raggedright\strut
HSQLDB \cite{lic:hsql} \strut
\end{minipage} & \begin{minipage}[t]{0.08\columnwidth}\raggedright\strut
2.3.4\strut
\end{minipage} & \begin{minipage}[t]{0.49\columnwidth}\raggedright\strut
Sistema gestor de base de datos.\strut
\end{minipage} & \begin{minipage}[t]{0.11\columnwidth}\raggedright\strut
BSD\strut
\end{minipage}\tabularnewline\tabularnewline


\begin{minipage}[t]{0.18\columnwidth}\raggedright\strut
Antlr \cite{lic:antlr} \strut
\end{minipage} & \begin{minipage}[t]{0.08\columnwidth}\raggedright\strut
2.7.7\strut
\end{minipage} & \begin{minipage}[t]{0.49\columnwidth}\raggedright\strut
\emph{Parser} para trabajar con texto en archivos binarios. \strut
\end{minipage} & \begin{minipage}[t]{0.11\columnwidth}\raggedright\strut
BSD\strut
\end{minipage}\tabularnewline\tabularnewline


\begin{minipage}[t]{0.18\columnwidth}\raggedright\strut
Commons Collections \cite{lic:col} \strut
\end{minipage} & \begin{minipage}[t]{0.08\columnwidth}\raggedright\strut
3.2.1\strut
\end{minipage} & \begin{minipage}[t]{0.49\columnwidth}\raggedright\strut
Framework que añade tipos de colecciones.\strut
\end{minipage} & \begin{minipage}[t]{0.11\columnwidth}\raggedright\strut
Apache v2.0\strut
\end{minipage}\tabularnewline\tabularnewline


\begin{minipage}[t]{0.18\columnwidth}\raggedright\strut
Commons Logging \cite{lic:colog} \strut
\end{minipage} & \begin{minipage}[t]{0.08\columnwidth}\raggedright\strut
1.1\strut
\end{minipage} & \begin{minipage}[t]{0.49\columnwidth}\raggedright\strut
Librería que permite generar logs de forma sencilla.\strut
\end{minipage} & \begin{minipage}[t]{0.11\columnwidth}\raggedright\strut
Apache v2.0\strut
\end{minipage}\tabularnewline\tabularnewline


\begin{minipage}[t]{0.18\columnwidth}\raggedright\strut
Dom4j \cite{lic:dom} \strut
\end{minipage} & \begin{minipage}[t]{0.08\columnwidth}\raggedright\strut
1.6.1\strut
\end{minipage} & \begin{minipage}[t]{0.49\columnwidth}\raggedright\strut
Framework XML flexible.\strut
\end{minipage} & \begin{minipage}[t]{0.11\columnwidth}\raggedright\strut
BSD\strut
\end{minipage}\tabularnewline\tabularnewline


\begin{minipage}[t]{0.18\columnwidth}\raggedright\strut
Hibernate \cite{lic:hiber} \strut
\end{minipage} & \begin{minipage}[t]{0.08\columnwidth}\raggedright\strut
Múltiples\strut
\end{minipage} & \begin{minipage}[t]{0.49\columnwidth}\raggedright\strut
Herramienta de mapeo Objeto-Relacional (ORM).\strut
\end{minipage} & \begin{minipage}[t]{0.11\columnwidth}\raggedright\strut
LGPL\strut
\end{minipage}\tabularnewline\tabularnewline


\begin{minipage}[t]{0.18\columnwidth}\raggedright\strut
Javassist \cite{lic:javss} \strut
\end{minipage} & \begin{minipage}[t]{0.08\columnwidth}\raggedright\strut
3.15.0\strut
\end{minipage} & \begin{minipage}[t]{0.49\columnwidth}\raggedright\strut
Librería para manipular bytecode Java.\strut
\end{minipage} & \begin{minipage}[t]{0.11\columnwidth}\raggedright\strut
LGPL\strut
\end{minipage}\tabularnewline\tabularnewline


\begin{minipage}[t]{0.18\columnwidth}\raggedright\strut
JBoss \cite{lic:jboss} \strut
\end{minipage} & \begin{minipage}[t]{0.08\columnwidth}\raggedright\strut
Múltiples\strut
\end{minipage} & \begin{minipage}[t]{0.49\columnwidth}\raggedright\strut
Servidor de aplicaciones Java EE.\strut
\end{minipage} & \begin{minipage}[t]{0.11\columnwidth}\raggedright\strut
LGPL\strut
\end{minipage}\tabularnewline\tabularnewline


\begin{minipage}[t]{0.18\columnwidth}\raggedright\strut
Log4j \cite{lic:log4j} \strut
\end{minipage} & \begin{minipage}[t]{0.08\columnwidth}\raggedright\strut
1.2.16\strut
\end{minipage} & \begin{minipage}[t]{0.49\columnwidth}\raggedright\strut
Biblioteca que permite escribir mensajes de registro.\strut
\end{minipage} & \begin{minipage}[t]{0.11\columnwidth}\raggedright\strut
Apache v2.0\strut
\end{minipage}\tabularnewline\tabularnewline


\begin{minipage}[t]{0.18\columnwidth}\raggedright\strut
MySQL Connector \cite{lic:mysql} \strut
\end{minipage} & \begin{minipage}[t]{0.08\columnwidth}\raggedright\strut
5.1.18\strut
\end{minipage} & \begin{minipage}[t]{0.49\columnwidth}\raggedright\strut
Proporciona drivers para JDBC.\strut
\end{minipage} & \begin{minipage}[t]{0.11\columnwidth}\raggedright\strut
LGPL \strut
\end{minipage}\tabularnewline\tabularnewline


\begin{minipage}[t]{0.18\columnwidth}\raggedright\strut
Javaee Web Api \cite{lic:jwa} \strut
\end{minipage} & \begin{minipage}[t]{0.08\columnwidth}\raggedright\strut
7.0\strut
\end{minipage} & \begin{minipage}[t]{0.49\columnwidth}\raggedright\strut
Framework que permite la creación de aplicaciones web Java.\strut
\end{minipage} & \begin{minipage}[t]{0.11\columnwidth}\raggedright\strut
GPL\strut
\end{minipage}\tabularnewline\tabularnewline


\begin{minipage}[t]{0.18\columnwidth}\raggedright\strut
Java Persistence API \cite{lic:jpa} \strut
\end{minipage} & \begin{minipage}[t]{0.08\columnwidth}\raggedright\strut
1.0.2\strut
\end{minipage} & \begin{minipage}[t]{0.49\columnwidth}\raggedright\strut
Framework que maneja datos relacionales en aplicaciones Java.\strut
\end{minipage} & \begin{minipage}[t]{0.11\columnwidth}\raggedright\strut
GPL\strut
\end{minipage}\tabularnewline\tabularnewline


\begin{minipage}[t]{0.18\columnwidth}\raggedright\strut
Selenium \cite{lic:sele} \strut
\end{minipage} & \begin{minipage}[t]{0.08\columnwidth}\raggedright\strut
3.4.0\strut
\end{minipage} & \begin{minipage}[t]{0.49\columnwidth}\raggedright\strut
Framework de automatización de pruebas en entorno web.\strut
\end{minipage} & \begin{minipage}[t]{0.11\columnwidth}\raggedright\strut
Apache v2.0\strut
\end{minipage}\tabularnewline\tabularnewline


\begin{minipage}[t]{0.18\columnwidth}\raggedright\strut
Guava \cite{lic:guava} \strut
\end{minipage} & \begin{minipage}[t]{0.08\columnwidth}\raggedright\strut
22.0\strut
\end{minipage} & \begin{minipage}[t]{0.49\columnwidth}\raggedright\strut
Conjunto de bibliotecas comunes para Java.\strut
\end{minipage} & \begin{minipage}[t]{0.11\columnwidth}\raggedright\strut
Apache v2.0\strut
\end{minipage}\tabularnewline\tabularnewline


\begin{minipage}[t]{0.18\columnwidth}\raggedright\strut
JUnit \cite{lic:junit} \strut
\end{minipage} & \begin{minipage}[t]{0.08\columnwidth}\raggedright\strut
4.11\strut
\end{minipage} & \begin{minipage}[t]{0.49\columnwidth}\raggedright\strut
Bibliotecas para realizar pruebas unitarias.\strut
\end{minipage} & \begin{minipage}[t]{0.11\columnwidth}\raggedright\strut
CPL\strut
\end{minipage}\tabularnewline


\bottomrule
\caption{Licencias del proyecto.}
\end{longtable}

Las licencias anteriormente citadas se caracterizan por \cite{lic:tipos}:

\begin{itemize}
\tightlist
\item
\textbf{Apache v2.0:} libre, abierta y con patentes. Solo es necesario avisar de que se está utilizando esta licencia. Permite al usuario usar el software, modificarlo, y distribuirlo, incluso si lo ha modificado.
\item
\textbf{BSD (Berkeley Software Distribution):} simple, libre y abierta, con clausula de advertencia.
\item
\textbf{EPL (Eclipse Public License):} libre, con patentes.
\item
\textbf{GLP (General Public License):} libre, abierta, con copyleft\footnote{El copyleft es una propiedad que impide que una versión modificada del software sea restrictiva, es decir la versión modificada de un software libre debe ser también libre.}. Garantiza al usuario la libertad de usar, compartir y modificar el software.
\item
\textbf{LGLP (Lesser General Public License):} GPL sin copyleft, permite enlazar con módulos no libres.
\end{itemize}

Teniendo en cuenta estas características y las características del proyecto que puede ser modificado en versiones futuras se establece una licencia GLP v3.0. Además es la propia recomendación del autor del framework jCOLIBRI.

Todas las licencias son compatibles con GLP \cite{lic:compgnu} excepto la licencias EPL utilizada por JUnit, por lo que el código que forman los test unitarios dispondrá de una licencia EPL.

\subsubsection{Documentación}

La documentación del proyecto está marcada por la descripción otorgada por el tutor del mismo. La licencia elegida para la documentación es \emph{Creative Commons} en la versión \emph{Reconocimiento-NoComercial-CompartirIgual 3.0 España (CC BY-NC-SA 3.0 ES)} que permite su evolución con reconocimiento de la autoría, impide el uso comercial de la obra y sus obras derivadas, las cuales deben distribuirse con una licencia igual a la que regula la obra original \cite{cc:ncsa}.

\subsubsection{Imágenes}

La mayoría de imágenes incluidas en la documentación no han sido obtenidas desde páginas de terceros. Sin embargo, alguna de las ilustraciones sí han sido obtenidas desde estas páginas.

Las imágenes adquiridas desde la página web \emph{Wikipedia}\footnote{MVC.png \cite{img:mvc}}, desde los apuntes de la Universidad de Burgos\footnote{MVC-Process.png \cite{img:mvc2}} y desde SlideShare\footnote{jsp-arch.png \cite{img:jsp2} y jsp-processing.png \cite{img:jsp1}} tienen licencia \emph{Creative Commons} con lo cual su uso es libre sin hacer uso comercial de las mismas.

Las imágenes obtenidas a partir del portal OpenI\footnote{MVCjsp.png \cite{img:repo}} tienen licencia \emph{Attribution 2.0 Generic (CC BY 2.0)} de \emph{Creative Commons}.

La ilustración adquirida desde el manual del framework jCOLIBRI\footnote{CBRcycle.png \cite{img:cbr}} está bajo licencia LGPL.

La imagen adquirida desde GitHub\footnote{scrum.png\cite{img:scrum}} tiene licencia \emph{OpenSource}.


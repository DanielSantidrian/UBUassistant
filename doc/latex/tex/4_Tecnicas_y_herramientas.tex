\capitulo{4}{Técnicas y herramientas}

\section{Metodología}\label{metodologia}

\subsection{Scrum}

Scrum es una metodología ágil para el desarrollo software que consiste en realizar de manera iterativa pequeños incrementos sobre una primera aproximación del producto \cite{scrum:info2}. 

Existen distintos componentes humanos en esta metodología \cite{scrum:info}

\begin{itemize}
\tightlist
\item
  \textbf{Product Owner:} se encarga de definir la funcionalidad y las prioridades en el product backlog junto con el scrum master.
\item
  \textbf{Scrum master:} coordina y soluciona problemas del equipo de desarrollo.
\item
  \textbf{Equipo de desarrollo:} encargado de la codificación. 
\end{itemize}

Como puede verse en la imagen \ref{fig:scrum} se diferencian distintos periodos y reuniones:

\begin{itemize}
\tightlist
\item
  \textbf{Sprint:} periodo de trabajo para desarrollar un incremento. La duración puede variar entre una y cuatro semanas.
\item
  \textbf{Reunión inicial:} se realiza al comienzo de cada sprint determinando el trabajo a realizar que se representa en el sprint backlog.
\item
  \textbf{Reunión diaria:} donde el equipo se autogestiona.
\item
  \textbf{Reunión final:} se realiza al final donde se revisa el producto obtenido.
\end{itemize}

\imagen{scrum}{Ciclo de la metodología scrum}

\subsection{Programación extrema}

Programación extrema (\emph{eXtreme Programming}) es una metodología de desarrollo ágil de software. Su característica principal es la adaptabilidad al cambio de requisitos en cualquier momento de la vida del proceso. Además, esta metodología tiene más propiedades tales como \cite{xp:wiki}:

\begin{itemize}
\tightlist
\item
  Desarrollo iterativo e incremental.
\item
  Integración frecuente del equipo de programación con el cliente.
\item
  Entregas frecuentes.
\item
  Corrección de errores antes de añadir nueva funcionalidad.
\item
  Código simple.
\end{itemize}

\section{Patrones de diseño}

\subsection{Modelo Vista Controlador}

Modelo vista controlador (\emph{MVC}) es un patrón arquitectónico que nos ayuda a separar los datos, la lógica de negocio y la interfaz de usuario \cite{mvc:wiki}.

\imagen{MVC}{Diagrama de clases MVC}

Sus componentes son los siguientes:

\begin{itemize}
\tightlist
\item
  \textbf{modelo:} representa los datos y la lógica de negocio.
\item
  \textbf{vista:} presenta la información del modelo.
\item
  \textbf{controlador:} controla las entradas del usuario y selecciona la vista.
\end{itemize}

\imagen{MVC-Process}{Diagrama de interacción MVC \cite{mvc:info}}

Aplicando estos conceptos a una aplicación web realizada mediante JSP con una base de datos como sistema de persistencia, podemos obtener el siguiente diagrama.

\imagen{MVCjsp}{Diagrama de MVC específico de una aplicación web JSP}

\section{Control de versiones}

\begin{itemize}
\tightlist
\item
  Herramientas consideradas: \href{https://git-scm.com/}{Git}.
\item
  Herramienta elegida: \href{https://git-scm.com/}{Git}.
\end{itemize}

Git es un software que permite gestionar los cambios producidos sobre los componentes de un proyecto. Se dice que es un gestor de versiones, definiendo versión como el estado de un proyecto en un determinado momento de su desarrollo. Git tiene diferentes características que lo hacían muy atractivo para este proyecto, como la gestión distribuida trabajando con copias locales \cite{git:wiki}.

\section{Hosting del repositorio}

\begin{itemize}
\tightlist
\item
  Herramientas consideradas: \href{https://bitbucket.org/}{Bitbucket}, \href{https://github.com/}{Github} y \href{https://xp-dev.com/}{XpDev}.
\item
  Herramienta elegida: \href{https://github.com/}{Github}.
\end{itemize}

GitHub es una plataforma donde poder alojar proyectos que utilizan Git como control de versiones. Si código es almacenado de forma pública la herramienta es gratuita.

GitHub dispone de una serie de características que favorecieron su elección en detrimentos de las demás herramientas consideradas. Alguna de estas características son la posibilidad de crear una página web del proyecto, ver gráficos, trabajo colaborativo e integración con herramientas de gestión de proyectos \cite{github:wiki}.

\section{Gestión del proyecto}

\begin{itemize}
\tightlist
\item
  Herramientas consideradas: Gestión manual.
\item
  Herramienta elegida: Gestión manual.
\end{itemize}

La gestión del proyecto fue manual, organizando cada semana las tareas pendientes, en curso y las terminadas, junto con las horas estimadas y reales.

\section{Comunicación}

\begin{itemize}
\tightlist
\item
  Herramienta considerada: email.
\item
  Herramienta elegida: email.
\end{itemize}

La comunicación entre los distintos integrantes de este proyecto ha sido realizada mediante email. El email proporciona una comunicación rápida y sencilla si los integrantes del proyecto son dos personas, como es el caso.

\section{Lenguaje de programación}
\begin{itemize}
\tightlist
\item
  Herramientas consideradas: \href{https://www.java.com/es/download/}{Java} y \href{https://www.python.org/}{Python}
\item
  Herramienta elegida: \href{https://www.java.com/es/download/}{Java}.
\end{itemize}

Java es un lenguaje de programación orientado a objetos, que soporta concurrencia y permite ejecutar un código sin tener que recompilarlo. Es uno de los lenguajes más usados, incluidas las aplicaciones web cliente-servidor \cite{java:wiki}.

\section{Entorno de desarrollo integrado (IDE)}

\begin{itemize}
\tightlist
\item
  Herramientas consideradas: \href{https://www.jetbrains.com/idea/}{IntelliJ}, \href{https://netbeans.org/}{NetBeans}, \href{http://gaia.fdi.ucm.es/research/colibri/colibristudio}{ColibriStudio} y \href{https://eclipse.org/}{Eclipse}.
\item
  Herramientas elegidas: \href{http://gaia.fdi.ucm.es/research/colibri/colibristudio}{ColibriStudio} y \href{https://eclipse.org/}{Eclipse}.
\end{itemize}

ColibriStudio es un IDE que integra todas las características del framework jCOLIBRI2. Este IDE nos permite crear aplicaciones basadas en el razonamiento basado en casos mediante asistentes visuales. En más alto nivel ColibriStudio es una versión modificada del popular Eclipse IDE \cite{colibri:studio}.

Eclipse es un entorno de desarrollo integrado compuesto por una serie de herramientas que permite el desarrollo de aplicaciones Java. Su versión JEE nos provee de las características necesarias para el desarrollo de una aplicación web JSP, pudiendo añadir software de terceros para ampliar su funcionalidad \cite{eclipse:info}.

Se empezó con el uso de ColibriStudio, pero una vez creado el núcleo de la aplicación CBR se pasó a Eclipse.

\section{Sistema de gestión de bases de datos}

\begin{itemize}
\tightlist
\item
  Herramientas consideradas: \href{http://www.postgresql.org.es/}{PostgeSQL} y \href{https://www.mysql.com/}{MySQL}.
\item
  Herramienta elegida: \href{https://www.mysql.com/}{MySQL}.
\end{itemize}

MySQL es uno de los sistemas gestor de base de datos más populares y usados gracias a la alta compatibilidad de los lenguajes de programación y su facilidad de uso \cite{mysql:wiki}.

\section{Documentación}

\begin{itemize}
\tightlist
\item
  Herramientas consideradas: \href{http://www.xm1math.net/texmaker/}{TexMaker} y \href{https://www.openoffice.org/es/}{OpenOffice}.
\item
  Herramienta elegida: \href{http://www.xm1math.net/texmaker/}{TexMaker}.
\end{itemize}

TexMaker es un editor de textos que agrupa las herramientas necesarias para crear un documento LaTeX.

LaTeX es un sistema de creación de documentos que permite el uso de comandos y macros para realizar documentos con una alta calidad tipográfica \cite{latex:wiki}.

\section{Servicios de integración continua}

\subsection{Cobertura del código}

\begin{itemize}
\tightlist
\item
  Herramienta considerada: \href{http://www.eclemma.org/}{EclEmma}.
\item
  Herramienta elegida: \href{http://www.eclemma.org/}{EclEmma}.
\end{itemize}

EclEmma es una herramienta Java que permite integrarse en Eclipse para analizar la cobertura del código respecto a los test unitarios. Representa la cobertura mediante colores sobre el propio código además de presentar un porcentaje detallado por clase \cite{eclemma:info}.

\subsection{Calidad del código}

\begin{itemize}
\tightlist
\item
  Herramientas consideradas: \href{https://sourceforge.net/projects/refactorit/}{RefactorIt}, InCode y \href{https://www.sonarqube.org/}{SonarQube}.
\item
  Herramientas elegidas: \href{https://sourceforge.net/projects/refactorit/}{RefactorIt}, InCode y \href{https://www.sonarqube.org/}{SonarQube}.
\end{itemize}

RefactorIt es una herramienta que puede integrarse en el IDE de Eclipse o utilizarse en su versión independiente y que proporciona diferentes métricas medidas sobre el código Java permitiendo, además, refactorizar automáticamente el mismo \cite{refacit:source}.

InCode, por su parte, es una herramienta que analiza en código Java en busca de defectos del código.

SonarQube es una plataforma que evalúa código fuente en búsqueda de código duplicado, cobertura de código, bugs y defectos de código \cite{wiki:sonar}.

\section{Sistemas de construcción automática del software}

\subsection{Maven}

Maven es una herramienta que se utiliza para la gestión y construcción de proyectos Java. Sus tareas referentes a la construcción son la compilación y el empaquetado. Mediante un Project Object Model (POM) con estructura XML, Maven, describe el proyecto a construir, sus dependencias y componentes externos \cite{maven:wiki}. 

\section{Sistemas de pruebas automáticas}

\subsection{Selenium}

Selenium es una herramienta que proporciona un entorno para desarrollar pruebas en entornos web. Selenium nos provee de un IDE para realizar grabaciones sobre una página web las pruebas deseadas. Una vez realizada la grabación nos permite exportar las pruebas a un lenguajes de programación determinado estando disponibles algunos como Java, Python, PHP... \cite{selenium:wiki}

\section{Librerías}

\subsection{JUnit}

JUnit lo forman un conjunto de bibliotecas utilizadas para realizar pruebas unitarias sobre el código \cite{junit:wiki}.

\subsection{jCOLIBRI2}

jCOLIBRI es un framework Open Source que permite la creación de sistemas basados en el razonamiento basado en casos (CBR). Proporciona todos los métodos y mecanismos para la creación de un sistema experto de la manera más simplificada posible.

Es un framework desarrollado en Java por la Universidad Complutense de Madrid y se distribuye bajo licencia GNU Lesser General Public License (LGPL) \cite{colibri:frame}.

\subsection{Google Guava}

Google Guava aes un conjunto de librerías donde se incluyen nuevos tipos de colecciones, colecciones inmutables, librerías de gráficos, y demás utilidades para aplicaciones Java \cite{guava:lib}.

\subsection{Selenium}

Selenium, como librería, otorga todas las herramientas para realizar pruebas automáticas sobre sitios web. Esta librería provee de todas las dependencias, como por ejemplo WebDriver, para poder realiza pruebas de interfaz en lenguaje Java \cite{sel:lib}.


\section{Desarrollo web}

\subsection{Servidor Apache Tomcat}

Apache Tomcat es un servidor que permite hospedar aplicaciones web desarrolladas en Java \cite{tomcat:wiki}.

\section{Otras herramientas}

\subsection{Mendeley}

Medeley es una aplicación que permite compartir y gestionar referencias bibliográficas, documentos, etc. Con su versión web y su versión de escritorio podemos añadir tanto referencias a páginas web, como a documentos que tengamos descargados \cite{mende:wiki}.

Una característica que la hizo interesante para su aplicabilidad es la capacidad de exportar las referencias guardadas en formato BibTex para poder utilizarse en LaTeX.
